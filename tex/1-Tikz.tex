% !TeX root = template-UNAMFC.tex

\section{Tikz y Beamer}
	\subsection{Tikz}
		\subsubsection{Flowboxes}

\begin{frame}{Hablamos de nuestro proyecto}{Con metas parciales, referencias y  resaltando  lo que hacemos nosotros}
    \begin{center}
    \begin{tikzpicture}[node distance=15mm and 15mm,font=\normalfont]
%        \setlength{\leftmargini}{10pt}

    % caract
        \node[flowbox] (caract) {%
            \fbtitle{Caracterización}\vphantom{yÖ}
         \nodepart{two}
            \begin{minipage}{.17\textwidth}
                \begin{itemize}\itemsep0em
                    \item Experimental
                    \item Teórica
                    \item Numérica
                \end{itemize}
            \end{minipage}
        };

    % quiestion
        \node[concept,above=of caract,align=center,yshift=-.6cm] (question) {
            Metasuperficie  plasmónica  desordenada
            };

        \draw[conn] (question) -- (caract);

    % teo
        \node[flowbox,right=of caract,yshift=0cm,onslide=<2->{fb-highlight}] (teo) {%
            \fbtitle{Teórica}\vphantom{yÖ}
         \nodepart{two}
            \begin{minipage}{.24\textwidth}
                \textit{Coherent Scattering Model\footnotemark[1]}\centering
                \begin{itemize}\itemsep0em  \small
                    \item Teoría de esparcimiento
                    \item Polidispersidad en tamaño
                    \item Partículas esféricas
                \end{itemize}
            \end{minipage}
        };

        \draw[conn] (caract) -- (teo);

    % num
        \node[flowbox,right=of teo,onslide=<2->{fb-highlight},xshift=0cm] (num) {%
            \fbtitle{Numérica}\vphantom{yÖ}
         \nodepart{two}
            \begin{minipage}{.20\textwidth}
                Descenso de gradiente\footnotemark[2]
                \begin{itemize}\itemsep0em
                    \item Ajuste a modelos
                    \item Minimizar error
                \end{itemize}
            \end{minipage}
        };

        \draw[conn] (teo) -- (num);

    % exp
        \node[flowbox,below=of teo] (exp) {%
            \fbtitle{Experimental}\vphantom{yÖ}
         \nodepart{two}
            \begin{minipage}{.24\textwidth}
                \begin{itemize}\itemsep0pt
                    \item Síntesis
                    \item Microscopía
                    \item Arreglo experimental\footnotemark[3]
                \end{itemize}
            \end{minipage}
        };


        % concept text nodes
        \node[concept, below right=of num,align=center,xshift=-1cm] (assump) {Biosensor};

        \draw[hidden] (question.east) -|
                      node[above,text=sdgrayFC,font=\itshape\scriptsize,anchor=south east] {Optimización/Condiciones experimentales}
                      (assump.30);%

        % connections
        {%
             \draw[conn] (exp.north -| teo.290) --
                        node[right,anchor=west,text=sdgrayFC,font=\itshape\scriptsize] {
                        \begin{minipage}{2cm}
                            \begin{flushleft}
                            Muestra
                            \end{flushleft}
                        \end{minipage}
                        }
                        (teo.290);
            \draw[conn] (teo.250) --
                        node[left,anchor=east,text=sdgrayFC,font=\itshape\scriptsize] {
                        \begin{minipage}{2cm}
                            \begin{flushright}
                            Diseño
                            \end{flushright}
                        \end{minipage}
                        }
                        (exp.north -| teo.250);
        }
        \draw[conn] (caract.south) |-
                        node[below,near end,anchor=north,text=sdgrayFC,font=\itshape\scriptsize] {
                        \begin{minipage}{2cm}
                            \begin{flushright}
                            Colaboración\\ INAOE
                            \end{flushright}
                        \end{minipage}
                        }
                        (exp.west);
        \draw[conn] (exp) -|
                        node[below,near start,anchor=north,text=sdgrayFC,font=\itshape\scriptsize] {Mediciones ópticas}
                        (num);
        \draw[conn] (num.300) |-
                        node[below,near end,anchor=north ,text=sdgrayFC,font=\itshape\scriptsize] {
                        \begin{minipage}{2cm}
                            \begin{center}
                            Análisis\\ sensitividad
                            \end{center}
                        \end{minipage}
                        }
                        (assump.west);
    \end{tikzpicture}
    \end{center}

    \begin{textblock*}{85mm}(170mm,115mm)
        \begin{flushleft}
        \begin{spacing}{0}\scriptsize%\fontsize{4}{12} \selectfont \boosterfont
            \footnotemark[1]\fullcite{vazquez-estrada_optical_2014}\\
            \footnotemark[2]\fullcite{barzilai_two-point_1988}\\
            \footnotemark[3]\fullcite{cuanalo_sensitivity_2022}
        \end{spacing}
        \end{flushleft}
    \end{textblock*}
\end{frame}



\begin{frame}{Algo semejante al anterior}
	{Pero este se muestra más limpio. Vale la pena leer el código de tikz}
	
	\begin{center}
		\begin{tikzpicture}[node distance=5mm and 60mm,font=\normalsize]
			%        \setlength{\leftmargini}{10pt}
			
			% mono
			\path (0,2) node[concept,align=center] (modelo) {
				Modelo teórico
			};
			\node[below=of modelo, font = {\normalsize}] {%
				\begin{minipage}{1.5cm}
					$\begin{aligned}
						&\vec{x} \equiv \text{Parámetros}\\
						&f_i(\vec{x}) \equiv \text{Modelo}\\
					\end{aligned}$
				\end{minipage}
			};
			
			\path (0,-5) node[concept,align=center] (exp) {
				Resultado experimental
			};
			\node[below=of exp,yshift=0cm, font = {\normalsize}] {%
				\begin{minipage}{1.5cm}
					$\begin{aligned}
						y_i(\vec{x}) &\equiv \text{Medición}\\
					\end{aligned}$
				\end{minipage}
			};
			
			\node[below right = of modelo, concept,align=center,yshift=-.5cm] (error) {
				Error};
			
			\node[below =of error,font = {\normalsize}](min){
				$\begin{aligned}
					F(\vec{x}) = \frac{1}{N}\sum_{i=1}^N\qty(f_i(\vec{x})-y_i)^2
				\end{aligned}$
			};
			
			\draw[hidden] (modelo.east) -| (error.north) ;
			\draw[hidden] (exp.east) -| (min.south) ;
			
			\node[flowbox,fb-highlight,right =of error,xshift=0cm,font = {\normalsize}] (des) {%
				\fbtitle{Descenso de gradiente}\vphantom{yÖ}
				\nodepart{two}
				\begin{minipage}{3.2cm}\centering
					$\begin{aligned}
						\vec{x}_{\ell+1} = \vec{x}_\ell - \gamma \nabla F(\vec{x}_\ell)
					\end{aligned}$
				\end{minipage}
			};
			
			\draw[hidden] (error.east) -- (des)
			node[right,near start,text=sdgrayFC,font=\itshape\normalsize,anchor=south] {
				Minimización
			};
			
			\node[flowbox,below=of des,yshift=-30mm,font = {\normalsize}] (barzilai) {%
				\fbtitle{Método de dos pasos}\vphantom{yÖ}
				\nodepart{two}
				\begin{minipage}{75mm}\centering
					$\begin{aligned}
						\gamma_\ell = \frac{\qty(\vec{x}_{\ell} - \vec{x}_{\ell-1}) \cdot  \qty[\nabla F(\vec{x}_\ell)-\nabla F(\vec{x}_{\ell-1})]}
						{\norm{\nabla F(\vec{x}_\ell)-\nabla F(\vec{x}_{\ell-1})}^2}
					\end{aligned}$
				\end{minipage}
			};
			
			\draw[conn] (des) -- (barzilai);
		\end{tikzpicture}
	\end{center}
	
	
	\begin{textblock*}{30mm}(185mm,80mm)
		\begin{flushright}\scriptsize
			\fullcite{barzilai_two-point_1988}
		\end{flushright}
	\end{textblock*}
	
\end{frame}


